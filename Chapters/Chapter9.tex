%Chapter 9

\chapter{Conclusion and future work} % Main chapter title

\label{future} % For referencing the chapter elsewhere, use \ref{future} 

\lhead{Conclusion and future work. \emph{Conclusion and future work}} % This is for the header on each page - perhaps a shortened title
%----------------------------------------------------------------------------------------
\section{Future work}
\subsection{Enhancing the semantic similarity module}
\subsubsection{Spreading activity}
Adding Spreading activity in wikipedia similarity module.
Spreading Activation is a technique that has been widely
adopted for associative retrieval (Crestani 1997). In associative
retrieval the idea is that it is possible to retrieve
relevant documents if they are associated with other
documents that have been considered relevant by the user.
In Wikipedia the links between articles show association
between concepts of articles and hence can be used as such
for finding related concepts to a given concept. The algorithm
starts with a set of activated nodes and in each iteration
the activation of nodes is spread to associated nodes.
The spread of activation may be directed by addition of
different constraints like distance constraints, fan out constraints,
path constraints, threshold etc. These parameters
are mostly domain specific.

\subsubsection{Probabilistic Latent Analysis (PLSA)}
PLSA models the probability of each co-occurrence as a mixture of conditionally independent multinomial distributions:
\begin{equation}
P(w,d) = \sum_{c} P(c)P(d|c)P(w|c)= P(d)\sum_{c}P(c|d)P(w|c)
\end{equation}
The first formulation is the symmetric formulation, where $w$ and $d$ are both generated from the latent class $c$ in similar ways (using the conditional probabilities $P(d|c)$ and $P(w|c)$), whereas the second formulation is the asymmetric formulation, where, for each document $d$, a latent class is chosen conditionally to the document according to $P(c|d)$, and a word is then generated from that class according to $P(w|c)$. Although we have used words and documents in this example, the co-occurrence of any couple of discrete variables may be modelled in exactly the same way.
So, the number of parameters is equal to . The number of parameters grows linearly with the number of documents. In addition, although PLSA is a generative model of the documents in the collection it is estimated on, it is not a generative model of new documents.

Other enhancements can be obtained by giving name entities(NER), titles, first statements of the article higher weight in the similarity.\newline
Or by using the database of Arabic wordNet(not the whole the project), as the existing modules that load from database have poor performance.

\subsection{Recommendation system}
\subsubsection{Na�ve Bayes Profile Model}
The model estimates the a posteriori probability, $P (c|d)$, of document $d$ belonging to class $c$. This estimation is based on the a priori probability, $P(c)$, the probability of observing a document in class $c$, $P (d|c)$, the probability of observing the document $d$ given $c$ and, $P (d)$, the probability of observing the instance $d$. Using these probabilities, the Bayes theorem is applied to calculate $P (c|d)$ as follows:
\begin{equation}
P(c|d) = \frac{P(c)P(d|c)}{P(d)}
\end{equation}
\subsection{System evaluation}
Evaluating more combinations of the used modules (Elkhoja stemmer, DBSCAN for clustering, wikipedia similarity module with feature reduction technique PCA)
\begin{itemize}
\item Using different clustering algorithms (CHAMELEON, hierarchical clustering).
\item Integrating the recommendation system with the social network.
\item Adding incremental clustering.
\end{itemize}
