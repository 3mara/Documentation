Chapter 4

\chapter{Tools} % Main chapter title

\label{Chapter4} % For referencing the chapter elsewhere, use \ref{Chapter1} 

\lhead{Chapter 4. \emph{Tools}} % This is for the header on each page - perhaps a shortened title

%----------------------------------------------------------------------------------------

\section{AWN}
Arabic WordNet is the Arabic analogue to the widely used WordNet for the English language. The Arabic WordNet (AWN) is a lexical database of the Arabic language following the development process of Princeton English WordNet and Euro WordNet. It utilizes the Suggested Upper Merged Ontology as an interlingua to link Arabic WordNet to previously developed wordnets. Christiane Fellbaum at Princeton was the project lead. The project was sponsored by DOI/REFLEX.\\
From http://www.globalwordnet.org/AWN/DataSpec.html you can get the XML data exchange specifications of the database. AWN contains about 11,000 synsets (including 1,000 NE).\\

There are several different ways for accessing the database:\\
\begin{itemize}
\item[1] The browser package (available at http://sourceforge.net/projects/awnbrowser/) includes the AWN data and Princeton WN2.0 mappings in a relational database. You can use the export facilities to export the data as XML or CSV to taylor them to your needs .\\
\item[2] The database can also be downloaded in XML format (linked to Princeton WN 2.0) from \url {http://nlp.lsi.upc.edu/awn/get_bd.php}\\
\item[3] A set of basic python functions for accessing the database can be obtained from: \url {http://nlp.lsi.upc.edu/awn/AWNDatabaseManagement.py.gz}\\
\end{itemize}
Functionality:\\
\begin{itemize}
\item AWN Browser: Browsing the database\\
\item AWN can be downloaded in XML format and access its content be directly used at developers' will.\\
\end{itemize}
Technology:\\
    Java, Perl, MySQL\\
Innovation:\\
	One of the most important lexical resources for Arabic language.\\

